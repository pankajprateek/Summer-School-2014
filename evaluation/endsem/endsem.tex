\documentclass[12pt]{article}
\usepackage{geometry}  
\usepackage{fancyhdr} % Required for custom headers
\usepackage{lastpage} % Required to determine the last page for the footer
\usepackage{extramarks} % Required for headers and footers
\usepackage{graphicx} % Required to insert images
%\usepackage{lipsum} % Used for inserting dummy 'Lorem ipsum' text into the template
\usepackage{enumitem}
\usepackage{amsmath}
\usepackage{amsfonts}
\usepackage{listings}
\usepackage{enumerate}
\usepackage{xcolor}
\geometry{body={6.5in, 9.0in},
  left=0.9in,
  top=1in
}

\lstset{language=C++,
  basicstyle=\ttfamily,
  tabsize=3,
  keywordstyle=\color{blue}\ttfamily,
  stringstyle=\color{green}\ttfamily,
  commentstyle=\color{red}\ttfamily,
  morecomment=[l][\color{magenta}]{\#},
  % backgroundcolor=\color{black!5},
  % basicstyle=\footnotesize,
}

\pagestyle{empty}
\date{ }
\begin{document}
\begin{center}
\textbf{\LARGE{ACA Summer School 2014}}\vspace{.5cm}

\large{Advanced C\texttt{++}} \\ \vspace{.5cm}

\textbf{\large{Endsem}}(Time: 1 hr; Max Marks: 50)
\end{center}
Name: \\
Roll number: 
\begin{itemize}
\item No weightage for writing extra/unnecessary code. Write only what is necessary.
\item Precise, clear and unambiguous answers will be appreciated.
\end{itemize}

\begin{enumerate}
\item Is something wrong in the following code? If no, what will be the output? If yes, what needs to be done to correct it? (3 marks)
  \begin{lstlisting}
#include<iostream>
using namespace std;

class color {
private:
  int red, blue, green;
public:
  color() {}
  color(int a, int b, int c) {red = a; blue = b; green = c;}
};
class item {
private:
  int a;
public:
  item(int x) {a = x;}
  item() {}
  friend void f(item x, color y);
};
void f(item x, color y) {
  cout<< x.a << endl;
  cout<< y.red << endl;
  cout<< y.green << endl;
  cout<< y.blue << endl;
}
int main() {
  item obj1(20);
  color white(255,255,255);
  f(obj1, white);
  return 0;
}
  \end{lstlisting}\vspace*{70mm}
\item 
  \begin{enumerate}[a)]
  \item Write a sample code to demonstrate run-time polymorphism. (5 marks)
  \item What calling mechanism is used while passing pointers? Call by value or call by reference? Please give reason for your claim. (3 marks)
  \end{enumerate}\vspace*{130mm}
\item Write the copy constructor for the following class: (7 marks)
  \begin{lstlisting}
class matrix {
private:
  int rows, cols;
int** mat;
public:
  matrix();
  matrix(int r, int c);
  ~matrix();
  void setVal(int r, int c, int val);
  int getVal(int r, int c);
};
  \end{lstlisting}
  Here, setVal() sets the value of cell (r,c) to val and getVal() returns the value in cell (r,c). Write only the copy constructor. Main etc is not needed.\vspace*{90mm}
\item What is the need for a virtual destructor? Explain and give a simple example to support your explanation. (7 marks)
\item \vspace*{60mm}What will be the output of the following code:(5 marks)
  \begin{lstlisting}
#include <iostream>
#include <vector>
using namespace std;
int main() {
  vector<int> myVec(5,25);
  iterator::vector<int> it;
  it = myVec.begin();
  myVec.insert(it, 10);
  it = myVec.begin()+2;
  myVec.erase(it);
  int sum = 0;
  for(it = myVec.begin(); it!= myVec.end(); it++) {
    sum = sum + (*it);
    cout<<`` ''<<*it;
  }
  cout<<endl;
  cout<<sum;
  return 0;
}
  \end{lstlisting}\newpage
\item What is multiple inheritance? Give an example for multiple inheritance. Just give the design of the classes. Do not define the functions. Assume that base class has a non-default constructor. (3+7 marks)\vspace*{75mm}
\item In the following code, for each derived class, state what variables/functions are present in their private, protected and public sections : (3+5 = 8 marks)
  \begin{lstlisting}
class base {
private:
  int a;
protected:
  float f1();
public:
  float b;
  void f2();
};
class derived1 : protected base {
private:
  char c;
protected:
  double d;
  double f3();
public:
  void f2(int x);
};
class derived2 : public derived1 {
private:
  long e;
public:
  void f4();
};
  \end{lstlisting}

\end{enumerate}

\end{document}
