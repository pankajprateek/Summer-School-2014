\documentclass{beamer}
\usepackage{amsmath}
\usepackage{graphicx}
\usepackage{url}
\usepackage{listings}
\usepackage{fancyvrb}
\usepackage[T1]{fontenc}
\usepackage{hyperref}
\usepackage{listings}
\mode<presentation>
%{ \usetheme{Warsaw} }
\title{ACA Summer School 2014\\ Advanced C\texttt{++}}

\setbeamercovered{transparent=5}

\author{Pankaj Prateek}
\institute{ACA, CSE, IIT Kanpur}
\date{\today}

\lstset{language=C++,
  basicstyle=\ttfamily,
  tabsize=3,
  keywordstyle=\color{blue}\ttfamily,
  stringstyle=\color{green}\ttfamily,
  commentstyle=\color{red}\ttfamily,
  morecomment=[l][\color{magenta}]{\#},
  backgroundcolor=\color{black!5},
  % basicstyle=\footnotesize,
}
 
\AtBeginSection[]  % "Beamer, do the following at the start of every section"
{
\begin{frame}<beamer> 
\frametitle{Outline} % make a frame titled "Outline"
\tableofcontents[currentsection]  % show TOC and highlight current section
\end{frame}
}

\begin{document}
%----------- titlepage ----------------------------------------------%
\begin{frame}
  \titlepage
\end{frame}

\begin{frame}[fragile]{Pointer: Basics}
  \begin{itemize}
    \item Variable
      \begin{center}
        Name/Identifier $\leftrightarrow$ Contains a value $\leftrightarrow$ Memory address
      \end{center}\pause
    \item Memory cells are numbered in continuation, giving every cell a unique number, which is called its address\pause
    \item Pointer
      \begin{center}
        Name/Identifier $\leftrightarrow$ Contains memory address of another variable
      \end{center}\pause
    \item Pointers contain the address of some other variable. They are said to ``point to'' that variable
  \end{itemize}
\end{frame}

\begin{frame}[fragile]{Pointer: Basics}
  \begin{lstlisting}
int a;
int *ptr = &a; //Reference operator
a = 10;
*ptr = 12;   //Dereference operator
  \end{lstlisting}
\end{frame}

\begin{frame}[fragile]{Pointer: Basics}
  \begin{itemize}
    \item Pointers are type-specific (that is there is a different pointer for every different type of variable)\pause
    \item Pointer to a pointer is also possible (double reference)\pause
    \item Extremely useful in call by reference mechanism\pause
    \item Also useful in dynamic memory allocation\pause
    \item Void pointer : a type-less pointer, but cannot be dereferenced without explicit type cast\pause
    \item Null pointer : A pointer that points to nothing or 0. Cannot be derefrenced. Doing that raises a run-time error/segmentation fault\pause
    \item Equivalent array names
  \end{itemize}
\end{frame}

\begin{frame}[fragile]{Pointers to structures}
  \begin{lstlisting}
struct temp {
  int a; float b;
};
int main() {
  struct temp obj1;
  struct temp * ptr;
  obj1.a = 10;
  obj1.b = 3.14;
  ptr = &obj1;
  cout << ptr -> a << endl;
  cout << (*ptr).b << endl;
}
  \end{lstlisting}
\end{frame}

\begin{frame}[fragile]{Pointer Arithmetic}
  \begin{itemize}
    \item ++ and $--$ operators are allowed on pointers\pause
    \item They correspond to advancing and retreating the pointer by the size of one object in memory\pause
    \item + and - are also allowed on pointers, which advance or retreat them suitably\pause
    \item *(a + 10) is equivalent to a[10] while (a + 10) is equivalent to \&a[10]\pause
    \item Should be used with caution. Can possibly be the result of array out of bounds errors, resulting in segmentation faults\pause
    \item Array names are treated as constant pointers. Operations like A = A+1 are invalid for them
  \end{itemize}
\end{frame}

\begin{frame}[fragile]{Pointer: Basics}
  \begin{itemize}
    \item References are alias (second name) to variables\pause
    \item They do not contain memory address like pointers do\pause
    \item Unlike pointers, once assigned, then cannot be changed later\pause
    \item There are no Null references. They have to be initialized when they are declared.\pause
    \item The actual pass by reference is implemented through them, passing pointers is still only pass by value
  \end{itemize}
\end{frame}

\begin{frame}[fragile]{Pointers to Objects}
  \begin{lstlisting}
class ABC {
  int var1;
public:
  setVar(int a);
  int getVar();
};
ABC obj1;
ABC* ptr1;
ptr1 = &obj1;
ptr1->setVar(20);
cout<<ptr1->getVar();
  \end{lstlisting}
\end{frame}

\begin{frame}[fragile]{Pointers to Objects}
  \begin{itemize}
    \item Obviously, private data members cannot be accessed through pointers.\pause
    \item Way to access the public members is exactly like that in structures\pause
    \item The type checking of pointers is slightly loose in some sense when it comes to classes
  \end{itemize}
\end{frame}

\begin{frame}[fragile]{Pointers to Objects of derived class}
  \begin{lstlisting}
class ABC {
  int var1;
public:
  setVar(int a);
  int getVar();
};
class XYZ: public ABC {
int var2;
public:
  void setVar2(int b);
  int getVar2();
};
ABC obj1;
ABC* baseptr;
XYZ obj2;
baseptr = &obj1;
baseptr = &obj2;
  \end{lstlisting}
\end{frame}

\begin{frame}[fragile]{Pointers to Objects of derived class}
  \begin{itemize}
    \item Derived class object also contains a part of itself as the Base class\pause
    \item Hence, a pointer to base class can also point to an object of the derived class\pause
    \item But such a pointer can point only to the part of the derived class which houses the base class\pause
    \item To access member of the derived class, we need an explicit pointer to the derived class\pause
    \item Even when using a base class pointer for a derived class object, only the inherited members are accessible and not the whole class
  \end{itemize}
\end{frame}

\begin{frame}[fragile]{The this pointer (revisited)}
  \begin{itemize}
  \item Suppose there is a class A, with a member function f()\pause
  \item When f() is called via some object of the class A, then inside body of f(), the keyword ``this'' stores the address of that object\pause
  \item The this pointer is passed as a hidden argument to all nonstatic member function calls\pause
  \item The this pointer is available as a ``local'' variable within the body of all nonstatic functions.\pause
  \item It is used to reference any member of the class if it is hidden due to some other local variable with same name
  \end{itemize}
\end{frame}

\begin{frame}[fragile]{Virtual}
  \begin{itemize}
  \item What we know : Its used to avoid ambiguity while inheriting from multiple classes\pause
  \item New thing : its applicable not only in inheritance, but also to indivisual class members\pause
  \item A member of a class that can be re-defined in its derived classes is called a virtual member\pause
  \item In order to declare a member of a class as virtual, we must precede its declaration with the keyword virtual
  \end{itemize}
\end{frame}

\begin{frame}[fragile]{Virtual}
  \begin{lstlisting}
class Polygon {
protected:
  int width, height;
public:
  void setVal(int w, int h);
  virtual float area();
};
  \end{lstlisting}\pause
  \begin{itemize}
    \item The area function can be ``redefined'' in the classes derived from polyon\pause
    \item Exactly same function signature is needed in that case
  \end{itemize}
\end{frame}

\begin{frame}[fragile]{Virtual}
  \begin{lstlisting}
class Polygon {
protected:
  int width, height;
public:
  void setVal(int w, int h) {
    width = w; height = h;
  }
  virtual float area() {return 0;}
};
class Rect: public Polygon {
public:
  float area() {return width*height;};
};
class Triangle: public Polygon {
public:
  float area() {return 0.5*width*height;};
};
  \end{lstlisting}
\end{frame}

\begin{frame}[fragile]{Pure Virtual Function}
  \begin{itemize}
    \item A function with the virtual keyword is called a virtual function\pause
    \item A function declared virtual can be kept undefined\pause
    \item Such undefined virtual functions are called ``Pure virtual functions\pause
      \begin{lstlisting}
        virtual float area() = 0;
      \end{lstlisting}\pause
      The zero at the end tells the compiler that area() is a pure virtual function
  \end{itemize}
\end{frame}

\begin{frame}[fragile]{Abstract Class: Concept}
  \begin{itemize}
  \item Its a program design concept\pause
  \item A class which is designed to be specifically used as a base class\pause
  \item No concrete objects are created out of it.\pause
  \item It only represents the common basic properties of the derived classes\pause
  \item It has at least one pure virtual function. (explained later)\pause
  \item Any class derived from an abstract class will also be an abstract class if the virtual function is not defined.
  \end{itemize}
\end{frame}

\begin{frame}[fragile]{Abstract Class}
  \begin{itemize}
  \item Earlier, abstract class was that class which was inherited virtually\pause
  \item Now : Abstract class is a class with at least one pure virtual function\pause
    \begin{lstlisting}
class Polygon {
protected:
  int width, height;
public:
  void setVal(int w, int h);
  virtual float area() = 0;
};
    \end{lstlisting}
  \end{itemize}
\end{frame}

\begin{frame}[fragile]{Abstract Class}
  \begin{itemize}
  \item There is at least one function without a definition. Hence no objects possible. Abstract class cannot be instantiated\pause
  \item However, we can create a pointer for an abstract class !!\pause
  \item Pointer to abstract class can still be used to point to objects of other classes
  \end{itemize}
\end{frame}

\begin{frame}[fragile]{Polymorphism}
  \begin{itemize}
  \item Pointer to a derived class is type-compatible with a pointer to its base class. \pause
  \item Polymorphism is the art of taking advantage of this simple but powerful and versatile feature\pause
  \item Use the pointer to abstract base class\pause
  \item Dynamically manipulate the objects of one of the derived classes
  \end{itemize}
\end{frame}

\begin{frame}[fragile]{Polymorphism: The CS example}
  \begin{itemize}
  \item Every player has some weapon to kill the enemy\pause
  \item Weapon can be used by left-click or right-click of the mouse\pause
  \item All weapons do the same function on moving the mouse\pause
  \item But each weapon has its own unique function on clicking the mouse\pause
  \item Base class : Weapon\pause
  \item Derived classes : Knife, Maverick, AK47, Magnum, DEagle ....
  \end{itemize}
\end{frame}

\begin{frame}[fragile]{Polymorphism: The CS Example}
  \begin{lstlisting}
class Weapon {
protected:
  int x,y;
public:
  void mouseMove(int x, int y);
  virtual void leftClick() = 0;
  virtual void rightClick() = 0;
};
  \end{lstlisting}\pause
  \begin{itemize}
  \item Every weapon moves the same way when the mouse is moved. This functionality is common to all weapons, and hence defined in the base class itself.\pause
  \item Every weapon has its own special functionality on a leftClick or a rightClick, and hence these weapon dependent features are left undefined in base class.\pause
  \item They will have to be defined properly in the actual derived class of every weapon
  \end{itemize}
\end{frame}

\begin{frame}[fragile]{Polymorphism: The CS Example}
  \begin{lstlisting}
class M4A1: public Weapon {
public:
  void mouseMove(int x, int y);
  virtual void leftClick() {
    // code to shoot a bullet
    // reload if no bullets
  }
  virtual void rightClick() {
    // Enable/Disable Silencer
  }
};
  \end{lstlisting}
\end{frame}

\begin{frame}[fragile]{Polymorphism: The CS Example}
  \begin{lstlisting}
int main() {
  Player comp = new Player();
  AK47 gun1();
  M4A1 gun2();
  // Player.gun is a pointer to the class weapon
  Player.gun = &gun1;
  // Calls to AK47
  Player.gun.leftClick();
  Player.gun.rightClick();

  Player.gun = &gun2;
  // Calls to M4A1
  Player.gun.leftClick();
  Player.gun.rightClick();
}
  \end{lstlisting}
\end{frame}

\begin{frame}[fragile]{Polymorphism: The CS Example}
  \begin{itemize}
    \item Player.gun is a pointer to class Weapon\pause
    \item Hence it can also be used to point to objects of AK47 and M4A1\pause
    \item True magic of polymorphism is that it invokes the correct function depending on the object being pointed to \pause
    \item In the example, it was never specified which type of weapon is being used.
  \end{itemize}
\end{frame}

\begin{frame}[fragile]{Virtual Destructors}
  \begin{lstlisting}
class Base {
public:
  Base() {cout<<''construct base''<<endl;}
  ~Base() {cout<<''Destroy base''<<endl;}
};
class Derive: public Base {
public:
  Dervie() {cout<<''construct derive''<<endl;}
  ~Derive() {cout<<''Destroy derive''<<endl;}
};
int main() {
  Base *baseptr = new Derive();
  delete basePtr;
}
  \end{lstlisting}
\end{frame}

\begin{frame}[fragile]{Virtual Destructors}
  \begin{itemize}
    \item Output\\
      Construct Base\\
      Construct Derive\\
      Destroy Base\pause
    \item The pointer to base class does not have access to the destructor of the derived class\pause
    \item Memory taken by members belonging to the derived class is not returned back to system.\pause
    \item Memory leak
  \end{itemize}
\end{frame}

\begin{frame}[fragile]{Virtual Destructors}
  \begin{lstlisting}
class Base {
public:
  Base() {cout<<''construct base''<<endl;}
  virtual ~Base() {cout<<''Destroy base''<<endl;}
};
class Derive: public Base {
public:
  Dervie() {cout<<''construct derive''<<endl;}
  ~Derive() {cout<<''Destroy derive''<<endl;}
};
int main() {
  Base *baseptr = new Derive();
  delete basePtr;
}
  \end{lstlisting}
\end{frame}

\begin{frame}[fragile]{Virtual Destructors}
  \begin{itemize}
    \item Output\\
      Construct Base\\
      Construct Derive\\
      Destroy Derive\\
      Destroy Base\pause
    \item Memory leak avoided by making destructor virtual, and hence accessible to base class pointer\pause
    \item Make destructor protected and non-virtual to deliberately avoid this
  \end{itemize}
\end{frame}

%%%%%%%%%%%%%%%%%%%%%%%%%%%%%%%%%%%%%%%%%%%%%%%%%%%%%%%%%%%%%%%%%%%%%%%%%%%%%%%%

%% \begin{frame}[fragile]{Templates}
%%   \begin{itemize}
%%     \item Generic Programming: style of programming which is free of types\pause
%%     \item Templates are foundations of generic programming\pause
%%     \item Templates allow us to write a generic function with types known dynamically (Remember function overloading!)
%%   \end{itemize}
%% \end{frame}

%% \begin{frame}[fragile]{Function Templates}
%%   \begin{lstlisting}
%% void PrintInt(int n) {
%%   cout<<''Data=''<<n<<endl;
%% }
%% void PrintFloat(float n) {
%%   cout<<''Data=''<<n<<endl;
%% }
%%   \end{lstlisting}\pause
%%   Apart from the input data type, both functions are identical
%% \end{frame}

%% \begin{frame}[fragile]{Function Templates}
%%   \begin{lstlisting}
%% Template<typename T>
%% void Print(T n) {
%%   cout<<''Data=''<<n<<endl;
%% }
%%   \end{lstlisting}\pause
%%   \begin{itemize}
%%     \item The first line tells the compiler what follows is a function template\pause
%%     \item Actual meaning of T, template-type parameter, would be deduced by the compiler from the arguement to the function
%%   \end{itemize}
%% \end{frame}

%% \begin{frame}[fragile]{Function Templates: Why???}
%%   \begin{itemize}
%%   \item The functions are generated by the compiler\pause
%%   \item The real benefit is that the programmer does not have to copy-paste the code, he does not need to write down a new overload for each new data-type which is created.\pause
%%   \item Most compilers make optimizations to the code from templates which would not be possible in their absence
%%   \end{itemize}
%% \end{frame}

%% \begin{frame}[fragile]{Class Templates}
%%   \begin{itemize}
%%   \item Helpful in defining types whose behaviour is generic and reusable\pause
%%     \begin{lstlisting}
%% class item {
%%   int data;
%%   ....
%% };
%% class item {
%%   float data;
%%   ....
%% };
%%     \end{lstlisting}\pause
%%   \item If similar functionality for other data-types is needed, need to duplicate code or maybe even entire class. It incurs code maintenance issues, increases code size at the source code as well as at binary level
%%   \end{itemize}
%% \end{frame}

%% \begin{frame}[fragile]{Class Templates}
%%   \begin{itemize}
%%   \item Instead define a class using a template\pause
%%     \begin{lstlisting}
%% template<typename T>
%% class item{
%%   T data;
%%   ....
%% };
%% item<int> item1;
%% item<float> item2;
%%     \end{lstlisting}\pause
%%   \item With function templates, the type of arguements were sufficient for the compiler to call the correct function, but with class templates, the template type should be explicitly passes in angle brackets
%%   \end{itemize}
%% \end{frame}

%% \begin{frame}[fragile]{Class Templates: Example}
%%   \begin{columns}
%%     \begin{column}{0.4\textwidth}
%% item<int> instantiated
%%       \begin{lstlisting}
%% item<int> item1;
%% item1.set(29);
%% item1.print();
%%       \end{lstlisting}\pause
%%     \end{column}
%%     \begin{column}{0.4\textwidth}
%% item<float> instantiated
%%       \begin{lstlisting}
%% item<float> item1;
%% item1.set(29.5);
%% item1.print();
%%       \end{lstlisting}\pause
%%     \end{column}
%%   \end{columns}
%%   \begin{itemize}
%%   \item No relation between the two instantiations. For the compiler and the linker, they are two different entities, say two different classes
%%   \end{itemize}
%% \end{frame}

%% \begin{frame}[fragile]{Templates: A note}
%%   \begin{itemize}
%%   \item Instead of keyword \texttt{typename} in the template definition, keyword \texttt{class} can also be used without any change in the sematics of the code\pause
%%     \begin{lstlisting}
%% template<typename T> 
%% // can be replaced by 
%% template<class T>
%%     \end{lstlisting}\pause
%%   \item I prefer typename to class because it shows that the template parameter need not be a class
%%   \end{itemize}
%% \end{frame}

%% \begin{frame}[fragile]{Library}
%%   \begin{itemize}
%%     \item In computer science, a library is a collection of subroutines or classes used to develop software quickly and efficiently\pause
%%     \item Usually libraries have functions/constructs which are used very frequently by programmers\pause
%%     \item Examples : STL, Boost C++\pause
%%     \item When you create a software, and want others to use it too, roll it out in the form of a library\pause
%%     \item How to use a particular library depends completely on the implementation of the library\pause
%%     \item Except STL, all other C++ libraries are usually non­standard and hence need to be installed separately from g++
%%   \end{itemize}
%% \end{frame}

%% \begin{frame}[fragile]{Standard Template Library}
%%   \begin{itemize}
%%     \item Simple repeatedly used procedures are needed to be coded by the programmer each time he wants to use them\pause
%%     \item For eg : Sorting, searching, string handling, etc\pause
%%     \item Some repeatedly used data structures are also needed to be developed from scratch each time\pause
%%     \item For eg : Stack, List, Linked List, Queue, Set, Tree, Heap, Map...\pause
%%     \item A secondary aim of C++ is also to enable the programmer to focus more on the larger picture, eliminating the details quickly\pause
%%     \item Hence, C++ comes with pre­defined fast, generic, template based collection of classes which are also very efficient
%%   \end{itemize}
%% \end{frame}

%% \begin{frame}[fragile]{Standard Template Library}
%%   \begin{itemize}
%%     \item The Standard Template Library (STL) is a C++ software library \pause
%%     \item The STL was created as the first library of generic algorithms and data structures for C++\pause
%%     \item Idea behind STL: generic programming, abstractness without loss of efficiency, the Von Neumann computation model\pause
%%     \item The STL achieves its results through the use of templates.\pause
%%     \item Modern C++ compilers are optimized to minimize any abstraction penalty arising from heavy use of the STL.\pause
%%     \item The STL provides a ready-made set of common classes for C++, that can be used with any built-in type and any user-defined type
%%   \end{itemize}
%% \end{frame}

%% \begin{frame}[fragile]{Standard Template Library}
%%   \begin{itemize}
%%   \item The Standard Template Library (STL) is composed of 4 parts:\pause
%%     \begin{itemize}
%%     \item Containers: Stack, Queue, List etc\pause
%%     \item Algorithms: Sort, Search, etc\pause
%%     \item Functors: Function Objects\pause
%%     \item Iterators: Random, Forward, etc
%%     \end{itemize}
%%   \end{itemize}
%% \end{frame}

%% \begin{frame}[fragile]{STL: Containers}
%%   \begin{block}{Deque}
%%     \begin{itemize}
%%     \item Double-ended queues are sequence containers with dynamic sizes that can be expanded or contracted on both ends\pause
%%     \item Unlike arrays, their size can change dynamically, with their storage being handled automatically by the container.\pause
%%     \item Unlike vectors, deques are not guaranteed to store all its elements in contiguous storage locations.\pause
%%     \item Thus deques do not allow direct access by offsetting pointers to elements like arrays or vectors.\pause
%%     \item The elements of a deque can be scattered in different chunks of storage. Memory is allocated in chunks to avoid over scattering.\pause
%%     \item Internally, more complex than vectors, but can be more efficient if the sequences are large
%%     \end{itemize}
%%   \end{block}
%% \end{frame}

%% \begin{frame}[fragile]{STL: Containers}
%%   \begin{block}{List}
%%     \begin{itemize}
%%     \item Sequence containers allowing constant time insert and erase operations within the sequence, and iteration in both directions.\pause
%%     \item List containers are implemented as doubly-linked lists\pause
%%     \item They are very similar to forward\_list: The main difference being that forward\_list objects are single-linked lists\pause
%%     \item The main drawback of lists is that they lack direct access to the elements by their position\pause
%%     \item Lists perform generally better in inserting, extracting and moving elements in any position
%%     \end{itemize}
%%   \end{block}
%% \end{frame}

%% \begin{frame}[fragile]{STL: Containers}
%%   \begin{block}{Stack}
%%     \begin{itemize}
%%     \item Implements the LIFO (Last In First Out) sequencing on the elements, with only one end for all insertion and extraction of data\pause
%%     \item Needs 2 arguments : Type of data, and type of container\pause
%%     \item Container is the type of the stack and should support the usual operations like push(), pop() etc\pause
%%     \item Standard containers of vector, deque and list can be used. By default, the standard container deque is used.\pause
%%     \item template < class T, class Container = deque<T> > class stack;
%%     \end{itemize}
%%   \end{block}
%% \end{frame}

%% \begin{frame}[fragile]{STL: Containers}
%%   \begin{block}{Maps}
%%     \begin{itemize}
%%     \item Maps are associative containers that store elements formed by a combination of a (key, value), following a specific order.\pause
%%     \item In a map, the key values are generally used to sort and uniquely identify the elements\pause
%%     \item While the mapped values store the content associated to this key.\pause
%%     \item A unique feature of maps is that they implement the operator[], which allows for direct access of the mapped value.\pause
%%     \item Maps are typically implemented as binary search trees
%%     \end{itemize}
%%   \end{block}
%% \end{frame}

%% \begin{frame}[fragile]
%% \frametitle{Sample Code}
%% \begin{lstlisting}
%%     #include<stdio.h>
%%     #include<iostream>
%%     // A comment
%%     int main(void)
%%     {
%%       printf(``Hello World\n'');
%%       return 0;
%%     }
%% \end{lstlisting}
%% \end{frame}

\end{document}
